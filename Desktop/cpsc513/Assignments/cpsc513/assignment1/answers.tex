\documentclass{article}

\usepackage{verbatim}
\usepackage{amsfonts}
\usepackage{amsmath}
\usepackage{enumitem}
\usepackage[margin=1in]{geometry}

\newcommand{\nat}{\mathbb{N}}
\newcommand{\lcm}{\mathrm{lcm}}

\begin{document}

\pagestyle{empty}

\title{CPSC 513 - Assignment 1 \\ Tutorial Section 1}
\author{
  \begin{minipage}{0.4\textwidth}
  \begin{flushleft}
  Isaac Azuelos\\
  isaac@azuelos.ca\\
  10019288
  \end{flushleft}
  \end{minipage}
  \begin{minipage}{0.4\textwidth}
  \begin{flushright}
  Stefan Knudsen \\
  sknudsen93@gmail.com \\
  10075761\\
  \end{flushright}
  \end{minipage}
}
\maketitle

\section*{Problem 1}

\begin{quote}
  Recall that if $n$ and $m$ are nonnegative integers, then the
  \textit{least common multiple} of $n$ and $m$, $\lcm(n, m)$, is the
  smallest integer that is divisible by both $n$ and $m$.

  \begin{enumerate}[label=\alph*)]
    \item What are the prime factorizations of $\lcm(n, m)$ and
      $\gcd(n, m)$?
    \item How large can $\lcm(n, m)$ be?
    \item What does the answer for (a) imply about the relationship
      between $\lcm(n, m)$ and $\gcd(n, m)$?
    \item Prove that the functions $\lcm(n, m)$ and $\gcd(n, m)$ of
      $n$ and $m$ are both \textit{primitive recursive} functions.
  \end{enumerate}
\end{quote}

\section*{Answer}
\begin{enumerate}[label=\alph*)]
\item By definition,
  $n = p_1^{e_1} \times p_2^{e_2} \times \cdots \times p_k^{e_k}$ and
  $m = p_1^{f_1} \times p_2^{f_2} \times \cdots \times p_k^{f_k}$,
  where $\forall i$, $p_i$ is a prime and $e_i, f_i \in \nat^+$.
  Then
  $$ 
    \gcd(n, m) = \prod_{i=1}^k p_i^{min(e_i,f_i)} 
    \text{ and } 
    \lcm(n,m) = \prod_{i=1}^k p_i^{max(e_i,f_i)}
  $$

\item At most, $\lcm(n, m)$ will be $mn$ since $m|mn$ and $n|mn$.

\item The product of $\lcm(n, m)$ and $\gcd(n, m)$ is always $mn$.

\item First, we will prove that $\lcm(m,n)$ is primitive recursive. We
  use an iterated product to calculate $\lcm(m,n)$.

  In lecture, it was shown that if $\varphi : \nat^{k+1} \to \nat$ is
  a primitive recursive function, then so is the iterated
  product $$\prod_{i=0}^{X_{k+1}} \varphi(x_1,...,x_k,i)$$. We set
  $k+1$ to max($m,n$), which is a primitive recursive function.

  We let $\varphi$ be the function that raises the $i$th prime to the
  power of max($e_i,f_i$), where $e_i,f_i$ are as defined above. As we
  have seen in lecture, finding the $i$th prime is a recursive
  function, and so is exponentiation, since $x^0 = 1$ and $x^{y+1} =
  x^y \cdot x$. Now, in the exponent is the function that computes
  max($e_i,f_i$), and we know that max is primitive recursive, as is
  the composition of primitive recursive functions. We just need to
  show that we can find $e_i$ and $f_i$ primitive recursively.

  As we saw in lecture, for the G\"odel encoding $[\alpha_1, \alpha_1,
    \cdots, \alpha_k] = z$, $\alpha_i = (z)_i$ is primitive recursive,
  where $1 \leq i \leq k$ and $\alpha_i$ is the $i$th prime. So the
  function $(x)_y$ returns the number of times that $x$ is found among
  $y$'s prime factors. To compute max($e_i,f_i$) then, we can compute
  max($(n)_i,(m)_i$).

  %% The function g, given two numbers $x$ and $y$, returns the number
  %% of times that $x$ is found among $y$'s prime factors. This will
  %% allow us to compute $e_i$ and $f_i$. The predicate ``divides''
  %% and the integer division function are primitive recursive
  %% functions. We know this because, as shown in ``Computability,
  %% Complexity and Languages'', $x|y$ ($x$ divides $y$) can be
  %% expressed, using bounded minimization, as $(\exists t)_{\leq x}
  %% (y \cdot t = x)$, and $\lfloor x / y \rfloor$ can be calculated
  %% by min$_{t \leq x}[(t+1) \cdot y > x]$. The function $g$ composes
  %% these functions so that $g(1,y) = 0$, $g(x,0) = 0$ (we can allow
  %% for two base cases by making an extra function), and $g(x,y) =
  %% x|y + (x|y * g(x,\lfloor y / x \rfloor))$.

  Since all of the functions that we have used are primitive
  recursive, their composition is also primitvive recursive and so
  $\lcm(m,n)$ is primitive recursive. We basically replaced
  $(z)_{t+1}$ with $max((m)_{t+1},(n)_{t+1})$ in a G\"odel encoding.
  Our iterated sum might go too far since it goes up to max($m,n$) and
  not $m$ or $n$'s highest prime factor, but this won't be a problem
  since adding trailing zeros in a G\"odel encoding doesn't change
  anything.

  Now we'll compute gcd. From part b, we know that $\gcd(m,n) =
  mn/lcm(m,n)$, where $\gcd(m,n), mn, \lcm(m,n) \in \nat$ and
  $\lcm(m,n) | mn$. Since multiplication and the quotient function are
  primitive recursive, we can calculate $\gcd(m,n)$ primitive
  recursively by computing $\lfloor$ $mn / \lcm(m,n) \rfloor$.
  However, this will be problematic when one of our numbers is 0. To
  solve this, we could add another base case at the bottom so that
  gcd($k,0$) = $k$. We could also compute gcd the same way we computed
  the lcm, except with max replaced with min.

\end{enumerate}

\section*{Problem 2}

\begin{quote}
  Recall that the \textit{Fibonacci numbers} $F_0, F_1, F_2, ...$ are
  defined using the following recurrence (or ``recursive definition'')
  for every $n \ge 0$,
  \begin{displaymath}
    F_n = \left\{
      \begin{array}{lr}
        0 & \mathrm{if}\ n = 0, \\ 1 & \mathrm{if}\ n = 1, \\ F_{n-1}
        + F_{n-2} & \mathrm{if}\ n \ge 2. \\
      \end{array}
    \right.
  \end{displaymath}
  Prove that the function $f : \nat \to \nat$ such that $f(n) = F_n$,
  for every integer $n \ge 0$, is primitive recursive.
\end{quote}

\section*{Answer}

Consider the following function $f$, where $f$ is defined as

$$f(X_1) = \ell(f'(X_1))$$

where $\ell$ is the left element projection of our paring function, as
given in the lectures. Clearly $f$ is defined through composition, and
is primitive recursive if $f'$ is primitive recursive.

We'll define the function $f'$ as

\begin{alignat*}{1}
  f'(0) &= \langle 0, 1 \rangle \\ f'(t+1) &= g(t, y)
\end{alignat*}

where $y = f'(t)$ and our $g$ is defined through composition to be

\begin{alignat*}{1}
  g(X_1, X_2) &= \langle r(u^2_2(X_1, X_2)), \ell(u^2_2(X_1, X_2)) +
  r(u^2_2(X_1, X_2)) \rangle \\ &= \langle r(X_2), \ell(X_2) + r(X_2)
  \rangle
\end{alignat*}

Since this is a direct application of the composition and primitive
recursive schemes, $f'$, and therefore $f$, are primitive recursive.

% Finally coming up with the exact way to state this to match our
% exact definitions of composition and primitive recursion took me
% *way* too long. :(

Now all that remains is to prove that $f(n) = F_n$, for all $n \ge 0$,
which we'll do by strong induction.

\subsection*{Base Cases}

We want to show that $f(n) = F_n$, for $n \in {0, 1}$ as our base cases.

\subsubsection*{Base Case 1: $f(0) = F_0$}

Directly from the definition of $F_n$, we have that $F_0 = 0$.  

From the definition of $f$, we have that
$f(0) = \ell(f'(0)) = \ell(\langle 0, 1\rangle) = 0$. So we know that
$f(0) = F_0$.

\subsubsection*{Base Case 2: $f(1) = F_1$}

Again, we get that $F_1 = 1$ directly from the definition of $F_n$.

Computing $f(1)$ takes a bit more effort.

\begin{alignat*}{1}
  f(1) &= \ell(f'(1)) \\
       &= \ell(g(0, f'(0)) \\
       &= \ell(g(0, \langle 0, 1 \rangle) \\
       &= \ell(\langle 
                 r(\langle 0, 1 \rangle), 
                 \ell(\langle 0, 1 \rangle) + r(\langle 0, 1 \rangle) 
               \rangle) \\
       &= \ell(\langle 1, 0 + 1 \rangle) \\
       &= 1
\end{alignat*}

But we can see that $f(1) = 1 = F_1$, which is what we wanted for this
base case.

\subsection*{Induction Step}

Let $k$ be an arbitrarily chosen integer such that $k > 1$. Using our
\textit{inductive hypothesis} that $f(i) = F_i$ for all
$0 \le i \le k$, we wish to show that $f(k+1) = F_{k+1}$.

First, we know that $F_{k+1} = F_{k} + F_{k-1} = f(k) + f(k-1)$ from
the definition of $F$ and our hypothesis. We'll try to manipulate
$f(k+1)$ to this form to show they're equivalent.

\begin{alignat}{1}
  f(k+1) &= \ell(f'(k+1)) \\
         &= \ell(g(k, f'(k))) \\
         &= \ell(\langle r(f'(k)), \ell(f'(k)) + r(f'(k)) \rangle)\\
         &= r(f'(k)) \\
         &= r(g(k-1, f'(k-1))) \\ 
         &= r( \langle r(f'(k-1)), \ell(f'(k-1)) + r(f'(k-1)) \rangle )\\
         &= \ell(f'(k-1)) + r(f'(k-1)) \\
         &= f(k-1) + r(f'(k-1)) \\ % \ell(f'(n)) == f(n) by def.
         &= f(k-1) + f(k) \\       % since f(k+1) == r(f'(k)), from above.
         &= f(k) + f(k-1)          % commutataive.
\end{alignat}

Which is exactly what we wanted. I think I need to explain a bit,
though, since the above is basically a wall of symbols. Most of the
work is just expanding on definitions. In step 3, we just evaluate the
outermost $\ell$, and in step 6 we're evaluating the outermost $r$. We
then simplify the terms, first using the definition of $f$ to simplify
the left term, and then using the previous result in this calculation
that $f(k+1) = r(f'(k)$, i.e. that $r(f'(k-1) = f(k)$, to simplify the
right hand of the addition. We then flip them around to get exactly
what we wanted, since addition if commutative.

So, by strong induction, we know that for all $k > 0$, $f(k) = F_k$.
We also have our previous result that $f$ is primitive
recursive. These are what we wanted.

\end{document}
